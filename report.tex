\documentclass[a4paper,11pt,twoside]{article}
%\documentclass[a4paper,11pt,twoside,se]{article}

\usepackage{UmUStudentReport}
\usepackage{verbatim}   % Multi-line comments using \begin{comment}
\usepackage{courier}    % Nicer fonts are used. (not necessary)
\usepackage{pslatex}    % Also nicer fonts. (not necessary)
\usepackage[pdftex]{graphicx}   % allows including pdf figures
\usepackage{listings}
\usepackage{pgf-umlcd}
%\usepackage{lmodern}   % Optional fonts. (not necessary)
%\usepackage{tabularx}
%\usepackage{microtype} % Provides some typographic improvements over default settings
%\usepackage{placeins}  % For aligning images with \FloatBarrier
%\usepackage{booktabs}  % For nice-looking tables
%\usepackage{titlesec}  % More granular control of sections.

% DOCUMENT INFO
% =============
\department{Department of Computing Science}
\coursename{Application Development in Java 7.5 p}
\coursecode{5DV135}
\title{Unit Test Framework}
\author{Lorenz Gerber ({\tt{dv15lgr@cs.umu.se}} {\tt{lozger03@student.umu.se}})}
\date{2016-11-17}
%\revisiondate{2016-01-18}
\instructor{Johan Eliasson / Jan Erik Moström / Alexander Sutherland / Filip Allberg / Adam Dahlgren Lindström}


% DOCUMENT SETTINGS
% =================
\bibliographystyle{plain}
%\bibliographystyle{ieee}
\pagestyle{fancy}
\raggedbottom
\setcounter{secnumdepth}{2}
\setcounter{tocdepth}{2}
%\graphicspath{{images/}}   %Path for images

\usepackage{float}
\floatstyle{ruled}
\newfloat{listing}{thp}{lop}
\floatname{listing}{Listing}



% DEFINES
% =======
%\newcommand{\mycommand}{<latex code>}

% DOCUMENT
% ========
\begin{document}
\lstset{language=C}
\maketitle
\thispagestyle{empty}
\newpage
\tableofcontents
\thispagestyle{empty}
\newpage

\clearpage
\pagenumbering{arabic}

\section{Usage} 
\subsection{Compiling from Command Line}
The source code for the Java application 'UnitTest' was divided into the packages `model', `view' and `controller'. The class containing the `main' method, the classes to be tested as well as the test classes where contained in the base directory (or default package). The following steps were conducted to build a jar application file:
\begin{verbatim}
mkdir build
javac -d ./build *.java controller/UnitTestController.java ./
    model/*.java view/*.java
cd build
jar cvfe UnitTest.jar UnitTestMain *.class controller/ view/ model/
\end{verbatim} 
Note that the `jar' file is named `UnitTest' while the `main' containing class is named `UnitTestMain'.
The source code is provided in a separate tar.gz file.

\subsection{Program Usage}
The `TestUnit' application is invoked from the command line by \verb+java -jar UnitTest+. Then the application GUI should open with a default test class `Test1' already chosen. To run the unit tests, the user presses the button `Run Tests' upon the output shows up in the main text area. An additional button under the main text area `clear' can be used to empty the main text output area. The application shall be terminated by closing the main application window using the system's window close button.

\section{System Description}

\section{Testing}


\addcontentsline{toc}{section}{\refname}
\bibliography{references}

\end{document}
